\part*{Jeugd}

\chapter{Oudkarspel} 

Ik ben op 20 februari 1928 geboren. We zijn alle drie thuis geboren met hulp van de vroedvrouw. Bert was van 1924 en Kees van 1930. Ik zat er tussenin.

\paragraph{}e

Vanaf 6 ging je lopend naar school. Het hoofd van de school woonde tegenover ons. Kwam Bert een keer thuis met een waterpistool; ze wilden op de meester schieten maar durfden het niet. Toen vroegen ze aan mij om dat te doen! Je weet niet hoe hard we daarna wegliepen! Toen kreeg m’n moeder de buurman aan de deur. Ze hadden een dochter en die had een eigen speelkamer.

\paragraph{}

Ik speelde veel met Kees vroeger. We konden goed samen spelen. Kees had ook wel zijn vriendjes en ik mijn vriendinnen. We hadden ringen in de bijkeuken hangen en een schommel buiten. 

\paragraph{}

Mijn moeder had toen al een wasmachine; die stond in de smederij. Want daar stond een slijpmachine en daar werd de wasmachine ook aangesloten. Er was een kooktoestel in de bijkeuken en in de winter zaten we daar. In het voorjaar gingen de mooie stoelen weer naar de voorkamer en aten we daar.

\paragraph{}

In de bijkeuken was ook een bedstee en daar sliepen de jongens.
Als klein kind sliep ik in de bedstee bij m’n ouders. Toen ik groter werd kreeg ik de kamer op zolder.
Eerst hadden we een WC buiten, achter de garage, boven het water. Maar toen ik jong was kregen we al een WC binnen. In Langedijk, als je daar ging varen, voer je de hele tijd onder de buiten Wc’s van de mensen door.
Iedere zondag gingen we naar het strand bij Camperduin. Dat was gezond vond m’n moeder. We hadden een grote tent mee waar je je kon verkleden. We gingen met de auto daarheen. Voor m’n vader was dat niet altijd zo leuk want hij was dan moe van het bedrijf en het was een heel gesjouw met al die spullen.

\paragraph{}

Bert was als kind altijd een beetje minnetjes. Veel ziek. Hij was heel veel op de boerderij van mijn grootouders in Schoorldam, dan was hij gelijk buiten. 
Mijn vader heeft toen een bokkenwagen gemaakt en opa hielp om de bok voor de kar te zetten.

We gingen ook vaak nog even aan in Schoorldam, bijna elke week.
En één keer zijn we met opa Vlam vanaf Schoorldam met een mooi rijtuig en een paard ervoor naar het strand geweest. Een rijtuig met van die kussens aan weerskanten. Op Camperduin kende hij mensen waar hij het paard kon stallen. Dat was een geweldige belevenis.

\paragraph{}

Mijn vader had een heel bedrijf; een smederij, een winkel en een grote garage. Daar stond de dorsmachine die door het hele dorp werd gehuurd. Als hij gebruikt werd stond hij op een veldje verderop en dan kwamen alle landbouwers met hun graan; dat kwam dan in zakken. 

\paragraph{}

Iedereen kwam met bootjes naar dat veld. Het land van de duizend eilanden heette het.  Iedereen had een veldje tussen het hoge water in. De bootjes werden vooruitbewogen met een kloet. Kees vond het ook prachtig om daarbij te zijn, het dorsen.
In de garage stonden ook een aantal auto’s; die werden verhuurd.
Het huis in Oudkarpsel is later gesloopt, het staat er nu niet meer. Mijn opa en oma, de ouders van mijn vader, woonden vlakbij. Zij woonden in een klein huisje achter ons huis. Een huisje met twee bedsteden.

Wij roeiden ook veel. Toen ik een keer bij de dokter was zei hij dat ik zulke sterke schouders had. Dat kwam van het roeien denk ik.

\chapter{Ouders en Voorouders}

Mijn ouders zijn tegelijkertijd getrouwd met haar broer, Piet Vlam en zijn vrouw. 
Met die broer had ze verder niet veel contact overigens. Ik vond het ook geen leuke man. Volgens mijn moeder was hij als kind wel leuk maar is hij zo geworden door zijn vrouw. Dat was een beetje een haaiebaai.

\paragraph{}
Mijn vader kwam uit Oudkarspel. Zijn ouders hadden daar de smederij in het huis waarin ik ben geboren. 


\paragraph{}
Zijn voorouders kwamen allemaal uit Noord-Holland; o.a. uit Enkhuizen. 
Hij had twee zussen. Die heetten Trien en Rentsje.
Hij was ondernemend en goed opgeleid, eerst tot smid, en later heeft hij in Antwerpen nog een opleiding m.b.t. auto’s gevolgd.

\paragraph{}

In hun verkeringstijd haalde hij mijn moeder op met de motor. Hij was de bink van de streek. Er was toen zo weinig verkeer dat ze hem al van verre hoorde aankomen. Mijn moeder was een Vlam uit Schoorldam. Zij had een broer Piet en een oudere zus, Hidda. 